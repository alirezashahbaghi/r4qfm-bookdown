\documentclass[]{book}
\usepackage{lmodern}
\usepackage{amssymb,amsmath}
\usepackage{ifxetex,ifluatex}
\usepackage{fixltx2e} % provides \textsubscript
\ifnum 0\ifxetex 1\fi\ifluatex 1\fi=0 % if pdftex
  \usepackage[T1]{fontenc}
  \usepackage[utf8]{inputenc}
\else % if luatex or xelatex
  \ifxetex
    \usepackage{mathspec}
  \else
    \usepackage{fontspec}
  \fi
  \defaultfontfeatures{Ligatures=TeX,Scale=MatchLowercase}
\fi
% use upquote if available, for straight quotes in verbatim environments
\IfFileExists{upquote.sty}{\usepackage{upquote}}{}
% use microtype if available
\IfFileExists{microtype.sty}{%
\usepackage{microtype}
\UseMicrotypeSet[protrusion]{basicmath} % disable protrusion for tt fonts
}{}
\usepackage[margin=1in]{geometry}
\usepackage{hyperref}
\hypersetup{unicode=true,
            pdftitle={R für quantitative Forschungsmethoden},
            pdfauthor={Alireza Shahbaghi},
            pdfborder={0 0 0},
            breaklinks=true}
\urlstyle{same}  % don't use monospace font for urls
\usepackage{natbib}
\bibliographystyle{apalike}
\usepackage{color}
\usepackage{fancyvrb}
\newcommand{\VerbBar}{|}
\newcommand{\VERB}{\Verb[commandchars=\\\{\}]}
\DefineVerbatimEnvironment{Highlighting}{Verbatim}{commandchars=\\\{\}}
% Add ',fontsize=\small' for more characters per line
\usepackage{framed}
\definecolor{shadecolor}{RGB}{248,248,248}
\newenvironment{Shaded}{\begin{snugshade}}{\end{snugshade}}
\newcommand{\KeywordTok}[1]{\textcolor[rgb]{0.13,0.29,0.53}{\textbf{#1}}}
\newcommand{\DataTypeTok}[1]{\textcolor[rgb]{0.13,0.29,0.53}{#1}}
\newcommand{\DecValTok}[1]{\textcolor[rgb]{0.00,0.00,0.81}{#1}}
\newcommand{\BaseNTok}[1]{\textcolor[rgb]{0.00,0.00,0.81}{#1}}
\newcommand{\FloatTok}[1]{\textcolor[rgb]{0.00,0.00,0.81}{#1}}
\newcommand{\ConstantTok}[1]{\textcolor[rgb]{0.00,0.00,0.00}{#1}}
\newcommand{\CharTok}[1]{\textcolor[rgb]{0.31,0.60,0.02}{#1}}
\newcommand{\SpecialCharTok}[1]{\textcolor[rgb]{0.00,0.00,0.00}{#1}}
\newcommand{\StringTok}[1]{\textcolor[rgb]{0.31,0.60,0.02}{#1}}
\newcommand{\VerbatimStringTok}[1]{\textcolor[rgb]{0.31,0.60,0.02}{#1}}
\newcommand{\SpecialStringTok}[1]{\textcolor[rgb]{0.31,0.60,0.02}{#1}}
\newcommand{\ImportTok}[1]{#1}
\newcommand{\CommentTok}[1]{\textcolor[rgb]{0.56,0.35,0.01}{\textit{#1}}}
\newcommand{\DocumentationTok}[1]{\textcolor[rgb]{0.56,0.35,0.01}{\textbf{\textit{#1}}}}
\newcommand{\AnnotationTok}[1]{\textcolor[rgb]{0.56,0.35,0.01}{\textbf{\textit{#1}}}}
\newcommand{\CommentVarTok}[1]{\textcolor[rgb]{0.56,0.35,0.01}{\textbf{\textit{#1}}}}
\newcommand{\OtherTok}[1]{\textcolor[rgb]{0.56,0.35,0.01}{#1}}
\newcommand{\FunctionTok}[1]{\textcolor[rgb]{0.00,0.00,0.00}{#1}}
\newcommand{\VariableTok}[1]{\textcolor[rgb]{0.00,0.00,0.00}{#1}}
\newcommand{\ControlFlowTok}[1]{\textcolor[rgb]{0.13,0.29,0.53}{\textbf{#1}}}
\newcommand{\OperatorTok}[1]{\textcolor[rgb]{0.81,0.36,0.00}{\textbf{#1}}}
\newcommand{\BuiltInTok}[1]{#1}
\newcommand{\ExtensionTok}[1]{#1}
\newcommand{\PreprocessorTok}[1]{\textcolor[rgb]{0.56,0.35,0.01}{\textit{#1}}}
\newcommand{\AttributeTok}[1]{\textcolor[rgb]{0.77,0.63,0.00}{#1}}
\newcommand{\RegionMarkerTok}[1]{#1}
\newcommand{\InformationTok}[1]{\textcolor[rgb]{0.56,0.35,0.01}{\textbf{\textit{#1}}}}
\newcommand{\WarningTok}[1]{\textcolor[rgb]{0.56,0.35,0.01}{\textbf{\textit{#1}}}}
\newcommand{\AlertTok}[1]{\textcolor[rgb]{0.94,0.16,0.16}{#1}}
\newcommand{\ErrorTok}[1]{\textcolor[rgb]{0.64,0.00,0.00}{\textbf{#1}}}
\newcommand{\NormalTok}[1]{#1}
\usepackage{longtable,booktabs}
\usepackage{graphicx,grffile}
\makeatletter
\def\maxwidth{\ifdim\Gin@nat@width>\linewidth\linewidth\else\Gin@nat@width\fi}
\def\maxheight{\ifdim\Gin@nat@height>\textheight\textheight\else\Gin@nat@height\fi}
\makeatother
% Scale images if necessary, so that they will not overflow the page
% margins by default, and it is still possible to overwrite the defaults
% using explicit options in \includegraphics[width, height, ...]{}
\setkeys{Gin}{width=\maxwidth,height=\maxheight,keepaspectratio}
\IfFileExists{parskip.sty}{%
\usepackage{parskip}
}{% else
\setlength{\parindent}{0pt}
\setlength{\parskip}{6pt plus 2pt minus 1pt}
}
\setlength{\emergencystretch}{3em}  % prevent overfull lines
\providecommand{\tightlist}{%
  \setlength{\itemsep}{0pt}\setlength{\parskip}{0pt}}
\setcounter{secnumdepth}{5}
% Redefines (sub)paragraphs to behave more like sections
\ifx\paragraph\undefined\else
\let\oldparagraph\paragraph
\renewcommand{\paragraph}[1]{\oldparagraph{#1}\mbox{}}
\fi
\ifx\subparagraph\undefined\else
\let\oldsubparagraph\subparagraph
\renewcommand{\subparagraph}[1]{\oldsubparagraph{#1}\mbox{}}
\fi

%%% Use protect on footnotes to avoid problems with footnotes in titles
\let\rmarkdownfootnote\footnote%
\def\footnote{\protect\rmarkdownfootnote}

%%% Change title format to be more compact
\usepackage{titling}

% Create subtitle command for use in maketitle
\newcommand{\subtitle}[1]{
  \posttitle{
    \begin{center}\large#1\end{center}
    }
}

\setlength{\droptitle}{-2em}
  \title{R für quantitative Forschungsmethoden}
  \pretitle{\vspace{\droptitle}\centering\huge}
  \posttitle{\par}
  \author{Alireza Shahbaghi}
  \preauthor{\centering\large\emph}
  \postauthor{\par}
  \predate{\centering\large\emph}
  \postdate{\par}
  \date{2018-04-02}

\usepackage{booktabs}
\usepackage{amsthm}
\makeatletter
\def\thm@space@setup{%
  \thm@preskip=8pt plus 2pt minus 4pt
  \thm@postskip=\thm@preskip
}
\makeatother

\usepackage{amsthm}
\newtheorem{theorem}{Theorem}[chapter]
\newtheorem{lemma}{Lemma}[chapter]
\theoremstyle{definition}
\newtheorem{definition}{Definition}[chapter]
\newtheorem{corollary}{Corollary}[chapter]
\newtheorem{proposition}{Proposition}[chapter]
\theoremstyle{definition}
\newtheorem{example}{Example}[chapter]
\theoremstyle{definition}
\newtheorem{exercise}{Exercise}[chapter]
\theoremstyle{remark}
\newtheorem*{remark}{Remark}
\newtheorem*{solution}{Solution}
\begin{document}
\maketitle

{
\setcounter{tocdepth}{1}
\tableofcontents
}
\chapter{Einleitung}\label{einleitung}

This is a \emph{sample} book written in \textbf{Markdown}. You can use
anything that Pandoc's Markdown supports, e.g., a math equation
\(a^2 + b^2 = c^2\).

The \textbf{bookdown} package can be installed from CRAN or Github:

\begin{Shaded}
\begin{Highlighting}[]
\KeywordTok{install.packages}\NormalTok{(}\StringTok{"bookdown"}\NormalTok{)}
\CommentTok{# or the development version}
\CommentTok{# devtools::install_github("rstudio/bookdown")}
\end{Highlighting}
\end{Shaded}

Remember each Rmd file contains one and only one chapter, and a chapter
is defined by the first-level heading \texttt{\#}.

To compile this example to PDF, you need XeLaTeX. You are recommended to
install TinyTeX (which includes XeLaTeX):
\url{https://yihui.name/tinytex/}.

\chapter{Skalenniveau}\label{skalenniveau}

\begin{itemize}
\tightlist
\item
  Nominal
\item
  Ordinal
\item
  Interval
\item
  Ratio
\end{itemize}

\section{Nominal}\label{nominal}

\begin{Shaded}
\begin{Highlighting}[]
\NormalTok{city <-}\StringTok{ }\KeywordTok{c}\NormalTok{(}\StringTok{"Berlin"}\NormalTok{, }\StringTok{"Hamburg"}\NormalTok{, }\StringTok{"Dresedn"}\NormalTok{, }\StringTok{"NY"}\NormalTok{)}

\NormalTok{country <-}\StringTok{ }\KeywordTok{c}\NormalTok{(}\StringTok{"Germany"}\NormalTok{, }\StringTok{"Germany"}\NormalTok{,}\StringTok{"Germany"}\NormalTok{, }\StringTok{"USA"}\NormalTok{)}

\NormalTok{gender <-}\StringTok{ }\KeywordTok{c}\NormalTok{(}\StringTok{"male"}\NormalTok{, }\StringTok{"male"}\NormalTok{, }\StringTok{"female"}\NormalTok{, }\StringTok{"male"}\NormalTok{)}
\end{Highlighting}
\end{Shaded}

\begin{Shaded}
\begin{Highlighting}[]
\NormalTok{city <-}\StringTok{ }\KeywordTok{c}\NormalTok{(}\StringTok{"Berlin"}\NormalTok{, }\StringTok{"Hamburg"}\NormalTok{, }\StringTok{"Dresedn"}\NormalTok{, }\StringTok{"NY"}\NormalTok{)}

\NormalTok{city }
\end{Highlighting}
\end{Shaded}

\begin{verbatim}
[1] "Berlin"  "Hamburg" "Dresedn" "NY"     
\end{verbatim}

\begin{Shaded}
\begin{Highlighting}[]
\KeywordTok{class}\NormalTok{(city)}
\end{Highlighting}
\end{Shaded}

\begin{verbatim}
[1] "character"
\end{verbatim}

\section{Ordinal}\label{ordinal}

\begin{Shaded}
\begin{Highlighting}[]
\NormalTok{rank <-}\StringTok{ }\KeywordTok{c}\NormalTok{(}\DecValTok{4}\NormalTok{, }\DecValTok{2}\NormalTok{, }\DecValTok{3}\NormalTok{, }\DecValTok{1}\NormalTok{, }\DecValTok{6}\NormalTok{, }\DecValTok{5}\NormalTok{)}

\NormalTok{temp <-}\StringTok{ }\KeywordTok{c}\NormalTok{(}\StringTok{"High"}\NormalTok{, }\StringTok{"Low"}\NormalTok{, }\StringTok{"High"}\NormalTok{, }\StringTok{"Medium"}\NormalTok{)}
\end{Highlighting}
\end{Shaded}

\begin{Shaded}
\begin{Highlighting}[]
\NormalTok{rank <-}\StringTok{ }\KeywordTok{c}\NormalTok{(}\DecValTok{4}\NormalTok{, }\DecValTok{2}\NormalTok{, }\DecValTok{3}\NormalTok{, }\DecValTok{1}\NormalTok{, }\DecValTok{6}\NormalTok{, }\DecValTok{5}\NormalTok{)}

\NormalTok{rank}
\end{Highlighting}
\end{Shaded}

\begin{verbatim}
[1] 4 2 3 1 6 5
\end{verbatim}

\begin{Shaded}
\begin{Highlighting}[]
\KeywordTok{class}\NormalTok{(rank)}
\end{Highlighting}
\end{Shaded}

\begin{verbatim}
[1] "numeric"
\end{verbatim}

\begin{Shaded}
\begin{Highlighting}[]
\NormalTok{rank <-}\StringTok{ }\KeywordTok{c}\NormalTok{(}\DecValTok{4}\NormalTok{, }\DecValTok{2}\NormalTok{, }\DecValTok{3}\NormalTok{, }\DecValTok{1}\NormalTok{, }\DecValTok{6}\NormalTok{, }\DecValTok{5}\NormalTok{)}

\NormalTok{rank.f <-}\StringTok{ }\KeywordTok{factor}\NormalTok{(rank)}

\NormalTok{rank.f}
\end{Highlighting}
\end{Shaded}

\begin{verbatim}
[1] 4 2 3 1 6 5
Levels: 1 2 3 4 5 6
\end{verbatim}

\begin{Shaded}
\begin{Highlighting}[]
\KeywordTok{class}\NormalTok{(rank.f)}
\end{Highlighting}
\end{Shaded}

\begin{verbatim}
[1] "factor"
\end{verbatim}

\begin{Shaded}
\begin{Highlighting}[]
\NormalTok{rank <-}\StringTok{ }\KeywordTok{c}\NormalTok{(}\DecValTok{4}\NormalTok{, }\DecValTok{2}\NormalTok{, }\DecValTok{3}\NormalTok{, }\DecValTok{1}\NormalTok{, }\DecValTok{6}\NormalTok{, }\DecValTok{5}\NormalTok{)}
\NormalTok{rank.f <-}\StringTok{ }\KeywordTok{factor}\NormalTok{(rank, }
                 \DataTypeTok{order =} \OtherTok{TRUE}\NormalTok{, }
                 \DataTypeTok{labels =} \KeywordTok{c}\NormalTok{(}\StringTok{"One"}\NormalTok{, }\StringTok{"Two"}\NormalTok{,}
                            \StringTok{"Three"}\NormalTok{, }\StringTok{"four"}\NormalTok{, }
                            \StringTok{"five"}\NormalTok{, }\StringTok{"six"}\NormalTok{))}
\NormalTok{rank.f}
\end{Highlighting}
\end{Shaded}

\begin{verbatim}
[1] four  Two   Three One   six   five 
Levels: One < Two < Three < four < five < six
\end{verbatim}

\begin{Shaded}
\begin{Highlighting}[]
\KeywordTok{class}\NormalTok{(rank.f)}
\end{Highlighting}
\end{Shaded}

\begin{verbatim}
[1] "ordered" "factor" 
\end{verbatim}

\begin{Shaded}
\begin{Highlighting}[]
\NormalTok{temp <-}\StringTok{ }\KeywordTok{c}\NormalTok{(}\StringTok{"High"}\NormalTok{, }\StringTok{"Low"}\NormalTok{, }\StringTok{"High"}\NormalTok{, }\StringTok{"Medium"}\NormalTok{)}

\NormalTok{temp}
\end{Highlighting}
\end{Shaded}

\begin{verbatim}
[1] "High"   "Low"    "High"   "Medium"
\end{verbatim}

\begin{Shaded}
\begin{Highlighting}[]
\KeywordTok{class}\NormalTok{(temp)}
\end{Highlighting}
\end{Shaded}

\begin{verbatim}
[1] "character"
\end{verbatim}

\begin{Shaded}
\begin{Highlighting}[]
\NormalTok{temp <-}\StringTok{ }\KeywordTok{c}\NormalTok{(}\StringTok{"High"}\NormalTok{, }\StringTok{"Low"}\NormalTok{, }\StringTok{"High"}\NormalTok{, }\StringTok{"Medium"}\NormalTok{)}

\NormalTok{temp.f <-}\StringTok{ }\KeywordTok{factor}\NormalTok{(temp)}

\NormalTok{temp.f}
\end{Highlighting}
\end{Shaded}

\begin{verbatim}
[1] High   Low    High   Medium
Levels: High Low Medium
\end{verbatim}

\begin{Shaded}
\begin{Highlighting}[]
\KeywordTok{class}\NormalTok{(temp.f)}
\end{Highlighting}
\end{Shaded}

\begin{verbatim}
[1] "factor"
\end{verbatim}

\begin{Shaded}
\begin{Highlighting}[]
\NormalTok{temp <-}\StringTok{ }\KeywordTok{c}\NormalTok{(}\StringTok{"High"}\NormalTok{, }\StringTok{"Low"}\NormalTok{, }\StringTok{"High"}\NormalTok{, }\StringTok{"Medium"}\NormalTok{)}
\NormalTok{temp.f <-}\StringTok{ }\KeywordTok{factor}\NormalTok{(temp,}
                 \DataTypeTok{order =} \OtherTok{TRUE}\NormalTok{,}
                 \DataTypeTok{labels =} \KeywordTok{c}\NormalTok{(}\StringTok{"Low"}\NormalTok{, }\StringTok{"Medium"}\NormalTok{, }\StringTok{"High"}\NormalTok{))}
\NormalTok{temp.f}
\end{Highlighting}
\end{Shaded}

\begin{verbatim}
[1] Low    Medium Low    High  
Levels: Low < Medium < High
\end{verbatim}

\begin{Shaded}
\begin{Highlighting}[]
\KeywordTok{class}\NormalTok{(temp.f)}
\end{Highlighting}
\end{Shaded}

\begin{verbatim}
[1] "ordered" "factor" 
\end{verbatim}

\section{Interval}\label{interval}

\begin{Shaded}
\begin{Highlighting}[]
\NormalTok{celsius <-}\StringTok{ }\KeywordTok{c}\NormalTok{(}\DecValTok{19}\NormalTok{ , }\DecValTok{20}\NormalTok{, }\DecValTok{23}\NormalTok{, }\DecValTok{27}\NormalTok{, }\DecValTok{35}\NormalTok{, }\DecValTok{30}\NormalTok{)}

\NormalTok{celsius}
\end{Highlighting}
\end{Shaded}

\begin{verbatim}
[1] 19 20 23 27 35 30
\end{verbatim}

\begin{Shaded}
\begin{Highlighting}[]
\KeywordTok{class}\NormalTok{(celsius)}
\end{Highlighting}
\end{Shaded}

\begin{verbatim}
[1] "numeric"
\end{verbatim}

\section{Ratio}\label{ratio}

\begin{Shaded}
\begin{Highlighting}[]
\NormalTok{kelvin <-}\StringTok{ }\KeywordTok{c}\NormalTok{(}\DecValTok{0}\NormalTok{ , }\DecValTok{273}\NormalTok{, }\DecValTok{200}\NormalTok{, }\DecValTok{350}\NormalTok{, }\DecValTok{354}\NormalTok{, }\DecValTok{300}\NormalTok{)}

\NormalTok{population <-}\StringTok{ }\KeywordTok{c}\NormalTok{(}\DecValTok{3610156}\NormalTok{, }\DecValTok{1774242}\NormalTok{, }\DecValTok{543825}\NormalTok{, }\DecValTok{8550405}\NormalTok{)}

\KeywordTok{class}\NormalTok{(kelvin)}
\end{Highlighting}
\end{Shaded}

\begin{verbatim}
[1] "numeric"
\end{verbatim}

\begin{Shaded}
\begin{Highlighting}[]
\KeywordTok{class}\NormalTok{(population)}
\end{Highlighting}
\end{Shaded}

\begin{verbatim}
[1] "numeric"
\end{verbatim}

\bibliography{packages.bib,book.bib}


\end{document}
